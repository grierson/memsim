\chapter{Methodology}
This project will utilise Lean principles to help maximise work flow while minimising possible risks. By implementing Lean principles, it will help identify issues early on and allow improvements to be made to prevent further problems.

Multiple management methods will be used within this project such as Kanban which will be used as the process management system to ensure features are being continuously implemented into the application, Behaviour-Driven Development will be used as the development process to ensure features meet their specifications, it will also be used to implement testing into the project as reassurance that the application is bug-free before shipping and lastly, it will be used to create testable documentation. 

Additional tools such as, Code Coverage will be used to check that all of the source code is covered by tests. Continuous integration will also be used to help protect the final product from new features which may break the application when merged into the project.

\section{Requirements Identified}

\section{Kanban}
A Kanban board will be used to visualise the work flow of the project to clearly depict the current state of the project to help distinguish what features of the project are being worked on and which tasks need completing. The advantage of using a Kanban board is that it will highlight delays within the current process allowing continuous improvements to be made to eliminate waste.

Requirements that add value to the product will be converted to User stories and be constantly added to the Backlog queue on the Kanban board throughout the entire project, allowing new features to added to the application while keeping the project flexible. The User stories will then be prioritised by most relevance to the project, so the feature that adds the most value to the product is always being worked on.

Kanban milestone - Estimation - Burndown Chart

Additionally, Kanban will also highlight issues with the current work-flow allowing small improvements to be made within the project to help improve productivity.

Benefits
Avoid Partially done work
Adding unessary features
Task Switching


\section{Behavior-Driven Development(BDD)}
BDD is comprised of several steps, firstly, an acceptance criteria is created using the Gherkin Language which describes the scenario of the intended feature and gives details on what should happen if a certain criteria is met. 
Once the acceptance criteria is created, a Step file is then made which is based of the acceptance criteria that will act as testable documentation for the software.

\subsection{Acceptance Test-Driven Development}
All features will have an Acceptance test written first before development is started on the feature. The advantages of writing an Acceptance test before creating the feature are firstly, that the Acceptance tests will ensure that the feature being implemented delivers the functionality that is expected. Secondly, it will provide human-readable documentation that is testable against the feature itself. Thirdly, It will highlight issues or bugs within the application which the user may experience before the project is shipped and lastly, it will provide reassurance that the delivered project is working correctly and meets the original specified specification.

By applying Acceptance tests to the project it will make each proposed feature testable so ensure the feature is performing the task as intended furthermore it will also work as a documentation of milestones within the project to demonstrate the feature has been implemented correctly due to it being able to test the code itself. "Acceptance tests check that your application works in the way the user will expected it to."\cite{sale2014testingpython}

\subsection{Unit Testing}
By writing tests beforehand it will increase the quality of the code and minimise bugs, it also ensures 
\section{Continuous integration}
\section{Code Coverage}